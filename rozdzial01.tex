\chapter{Wstęp}
Globalny dostęp do internetu w szybkim tempie zaczyna wypierać technologie desktopowe na rzecz aplikacji internetowych. W związku z tym coraz więcej przedsiębiorstw decyduje się na taką implementację swoich rozwiązań oraz produktów.

Zaletami aplikacji webowych są minimalne wymagania ze strony potencjalnego klienta oraz prostszy sposób implementacji. Od konsumenta wymagane jest jedynie urządzenie z dostępem do internetu. Taką aplikację można uruchomić na każdym urządzeniu codziennego użytku -- od komputerów stacjonarnych aż po urządzenia mobilne. Od strony implementacyjnej należy zapewnić serwer z zasobami pozwalającymi na komfortowe korzystanie z danego serwisu. Zasoby są ustalane na etapie analizy systemowej, a w trakcie rozwoju aplikacji, takie zasoby mogą być rozszerzane.

% Zastosowanie rozwiązania webowego pozwala na dotrarcie z produktem do szerszej grupy konsumentów, co przyczynia się do zwięszenia zysków.
W ciągu ostanich kilku lat znacząco zmieniło się podejście do tworzenia aplikacji internetowych. Jeszcze kilka lat temu dużą popularnością cieszyło się podejście do tworzenia aplikacji wielostronicowych (\textit{ang. MPA - Multi-Page application}). Oznacza to, że plik HTML był generowany po stronie backendu a następnie wysyłany do użytkownika. Takie podejście charateryzowało się ciągłą potrzebą przeładowywania strony. Obecnie w wyniku wzrostu popularności języka \textit{JavaScript}, pojawiło się niezliczona ilosć frameworków oferujących tworzenie aplikacji jednostronicowych (\textit{ang. SPA - Single-page application}). Cała logika stoi po stronie przeglądarki i na bieżąco aktualizuje swoją zawartość. Zaletą tego rozwiązania jest zrezygnowanie z widoku, generowaniego po stronie backendu co znacznie poprawia wydajność oraz szybkość aplikacji. Takie cechy serwisów są obecnie bardzo pożądane przez fakt wzrostu wymagań dotyczących pozytywnych doświadczeń użytkownika płynących z korzystania z aplikacji



% Globalny dostęp do Internetu spopularyzował rozpowszechnianie się aplikacji oraz serwisów internetowych. Obecnie coraz więcej przedsiębiorstw decyduje się na implementację swoich rozwiązań w takiej postaci. Zaletą takich rozwiązań są minimalne wymagania sprzętowe użytkownika - aplikacja stoi na zewnętrznym serwerze, który komunikuje się z klientem za pośrednictwem protokołu http, zatem od użytkownika wymagane jest urządzenie z dostępem do internetu.

% Zastosowanie aplikacji internetowych pozwala zatem dotrzeć z potencjalnym produktem do większej liczby konsumentów bez konieczności instalowania dodatkowego oprogramowania. Wystarczą urządzenia, które użytkujemy na co dzień -- od komputerów aż po smartfony.

Obecnie dużą popularnością cieszy się podejście do tworzenia tzw. Single-Page Application. Czyli aplikacji, których głównym założeniem jest wysyłanie do użytkownika jednego pliku HTML, który na bieżąco podmienia swoją zawartość z poziomu przeglądarki. W trakcie korzystania z programów, strony internetowe nie są przeładowywane, co powoduje szybsze działanie aplikacji oraz pozytywnie wpływa na doświadczenia użytkowników. W przypadku starego podejścia, tzn Multi-Page Application, pliki były generowane na bieżąco po stronie serwera i wysyłane do przeglądarki co znacząco wydłużało czas ładowania strony.
\section{Cel oraz zakres pracy}
Celem pracy jest zaprojektowanie i zaimplementowanie serwisu internetowego w oparciu o  nowoczesne technik programowania, którego zadaniem jest wspomaganie planowania wyprowadzania psów na spacer. Aplikacja skonstruowana jest w oparciu o dwie główne grupy docelowych użytkowników -- właścicieli oraz opiekunów. 

Zakres pracy obejmuje stworzenie aplikacji przy użyciu języka Java do stworzenia serwera oraz aplikacji dostępowej stworzonej przy pomocy frameworka Angular. Serwis pobiera i zapisuje dane przy użyciu MongoDB -- nowoczesnej oraz nierelacyjnej bazie danych, która w ciągu ostatnich lat zdobywa coraz większą popularność. Wykorzystano również szereg narzędzi wspomagających i usprawniających tworzenie aplikacji.

\section{Główne założenia projektowe}
\begin{itemize}
    \item Właściciele dostają szereg funkcjonalności, pozwalających na stworzenie profilu zwierzaka oraz zaplanowanie spacerów z deklaracją godziny oraz miejsca spaceru. Możliwe jest również wystawianie opinii po spacerze w celu weryfikacji kompetencji opiekuna;
    \item Opiekunowie otrzymują możliwość wyszukiwania spacerów, która bazuje na lokalizacji użytkownika. Aplikacja udostępnia tygodniowy oraz miesięczny wygląd planera ze spacerami, na które użytkownik się zapisał. Opiekunowie również mogą dodawać informacje o odbytym spacerze;
    \item Administracja dostaje możliwość podglądu bazy użytkowników, zwierząt oraz spacerów, jak i monitorowania samej aplikacji oraz błędów, które zgłosili inni uzytkownicy;
\end{itemize}
\section{Układ pracy}
Praca została podzielona na niżej wymienione rozdziały:
\begin{itemize}[leftmargin=1cm]
    \item Część teoretyczna, opisuje dokładnie wykorzystane technologie, techniki oraz programy, które zostały wykorzystane przy implementacji pracy;

    \item Część implementacyjna, skupia się na szczegółach dotyczących implementacji aplikacji oraz zawiera opis funkcjonalny;

    \item Podsumowanie, streszcza proces tworzenia aplikacji, oraz prezentuje dalsze możliwości rozwoju aplikacji w przyszłości;
\end{itemize}

% Praca dyplomowa
% \begin{itemize}
%         \item Wstęp $\frac{1}{3}$ strony
%         \item Cel pracy $\frac{1}{3}$ strony
%         \item Zakres pracy $\leftarrow$ co jest w pracy
%         \item Część teoretyczna: wprowadzenie, opis, technologie
%         \item Część implementacyjna: szczegóły implementacyjne, opis funkcjonalny
%         \item Podsumowanie
%         \item Spis tabel, rysunków, literatury
% \end{itemize}
