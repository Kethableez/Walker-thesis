\chapter{Część teoretyczna}
\section{Wprowadzenie do projektu}
Genezą powstania serwisu internetowego jest fakt, że istnieje duża grupa osób z psami, które z różnych przyczyn nie mają czasu na regularne spacery ze swoim pupilem. Dodatkową funkcjonalnością oferowaną przez aplikację jest możliwość wyprowadzania potrzebujących psów na spacer jako opiekun. Jest to opcja przeznaczona dla osób, które lubią zajmować się zwierzętami i mają więcej wolnego czasu, niż opiekunowie.

\section{Proces}
Proces tworzenia aplikacji obejmował kilka etapów:
\begin{itemize}
    \item Opracowanie wymagań funkcjonalnych oraz niefunkcjonalnych
    \item Opracowanie widoków aplikacji
    \item Opracowanie warstwy biznesowej po stronie backendu
    \item Opracowanie warstwy dostępowej po stronie frontendu
\end{itemize}

Każdy z etapów był zarządzany przy pomocy aplikacji Trello. Jest to darmowa aplikacja pozwalająca na skuteczne zarządzanie projektem przez jedną lub wiele osób współpracujacych ze sobą. Głównym zamysłem było stworzenie środowiska
\section{Wymagania funkcjonalne oraz niefunkcjonalne}
W pierwszym etapie projektowania serwisu został stworzony spis wymagań funkcjonalnych oraz niefunkcjonalnych. Miało to na celu zorganizowanie zadań na odpowiednie podetapy. i tak dalej
\subsection{Wymagania funkcjonalne}
\begin{itemize}
    \item Rejestracja nowych użytkowników z podziałem na role
    \item Logowanie użytkowników do serwisu
    \item Możliwość zmiany danych użytkownika - zdjęcie profilowe, dane osobowe, hasło
    \item Możliwość dodania profilu swojego psa
    \item Możliwość stworzenia spaceru
    \item Ocena opiekunów po zakończonym spacerze
    \item Przeglądanie listy dostępnych spacerów
    \item Przeglądanie listy nadchodzących spacerów
    \item Zapisywanie na spacery
    \item Wypisywanie się ze spacerów
    \item Dodawanie opinii o wyprowadzonym zwierzaków
    \item Przeglądanie historii spacerów
    \item Przeglądanie profili uzytkowników
    \item Przeglądanie profili psów
    \item Przeglądanie profili spacerów
    \item Zgłaszanie błędów w aplikacji przez użytkowników
    \item Zarządzanie kontami użytkowników - blokowanie, banowanie
    \item Wyświetlanie bazy danych użytkowników
    \item Wyświetlanie bazy danych zwierzaków
    \item Wyświetlanie bazy danych spacerów
    \item System zarządzania zgłoszonymi błędami
\end{itemize}
\subsection{Wymagania niefunkcjonalne}
\begin{itemize}
    \item Dostęp do internetu
\end{itemize}
\section{Diagram oraz przypadki użycia}

Przypadki użycia:
\newline
\textbf{Przypadek użycia:} Rejestracja \\
\textbf{Aktor:} Niezalogowany użytkownik \\
\textbf{Opis:} Rejestracja w serwisie\\
\textbf{Warunki wstępne:} Użytkownik niezalogowany, nieposiadający konta, wchodzący do serwisu po raz pierwszy\\
\textbf{Przebieg:}
\begin{itemize}
    \item Użytkownik klika w odnośnik "..."
    \item Użytkownik zostaje przeniesiony do strony z formularzem rejestracyjnym
    \item  Użytkownik wypełnia dane
    \begin{itemize}
        \item Dane są niepoprawne - serwis informuje uzytkownika o błędach, przycisk jest nieaktywny
        \item Dane są poprawne - przycisk jest aktywny i użytkownik może stworzyć konto
    \end{itemize}
    \item Wysłanie formularzu - użytkownik jest informowany o sukcesie bądź błędzie operacji
\end{itemize}

\textbf{Przypadek użycia:} Logowanie\\
\textbf{Aktor:} Niezalogowany użytkownik\\
\textbf{Opis:} Logowanie do serwisu\\
\textbf{Warunki wstępne:} Użytkownik niezalogowany, posiadający konto w serwisie\\
\textbf{Przebieg:}
\begin{itemize}
    \item Użytkownik wypełnia formularz logowania
    \item Użytkownik klika w przycisk "Zaloguj się"
    \begin{itemize}
        \item W przypadku sukcesu użytkownik zostaje przekierowany do strony aplikacji
        \item W przypadku błędu użytkownik jest informowany o błędzie
    \end{itemize}
\end{itemize}

\textbf{Przypadek użycia:} Zmiana zdjęcia profilowego\\
\textbf{Aktor:} Zalogowany użytkownik\\
\textbf{Opis:} Zmiana zdjęcia profilowego\\
\textbf{Warunki wstępne:} Użytkownik zalogowany z aktywnym kontem\\
\textbf{Przebieg:}
\begin{itemize}
    \item Użytkownik wchodzi w zakładkę profil
    \item Po najechaniu na zdjęcie profilowe klika w ikonę edycji
    \item Użytkownik wgrywa nowe zdjęcie
    \item Użytkownik klika w przycisk "Zmień zdjęcie"
    \item Użytkownik otrzymuje informację zwrotną o statusie wykonanej operacji
\end{itemize}

\textbf{Przypadek użycia:} Zmiana opisu\\
\textbf{Aktor:} Zalogowany użytkownik\\
\textbf{Opis:} Zmiana opisu profilu\\
\textbf{Warunki wstępne:} Użytkownik zalogowany z aktywnym kontem\\
\textbf{Przebieg:}
\begin{itemize}
    \item Użytkownik wchodzi w zakładkę profil
    \item Po najechaniu na opis profilowe klika w ikonę edycji
    \item Użytkownik wypełnia formularz
    \item Użytkownik klika w przycisk "Zmień opis"
    \item Użytkownik otrzymuje informację zwrotną o statusie wykonanej operacji
\end{itemize}

\textbf{Przypadek użycia:} Zmiana danych\\
\textbf{Aktor:} Zalogowany użytkownik\\
\textbf{Opis:} Zmiana danych\\
\textbf{Warunki wstępne:} Użytkownik zalogowany z aktywnym kontem\\
\textbf{Przebieg:}
\begin{itemize}
    \item Użytkownik wchodzi w zakładkę profil
    \item Po najechaniu na dane klika w ikonę edycji
    \item Użytkownik wypełnia formularz
    \item Użytkownik klika w przycisk "Zmień dane"
    \item Użytkownik otrzymuje informację zwrotną o statusie wykonanej operacji
\end{itemize}

\textbf{Przypadek użycia:} Zmiana hasła\\
\textbf{Aktor:} Zalogowany użytkownik\\
\textbf{Opis:} Zmiana hasła\\
\textbf{Warunki wstępne:} Użytkownik zalogowany z aktywnym kontem\\
\textbf{Przebieg:}
\begin{itemize}
    \item Użytkownik wchodzi w zakładkę profil
    \item Po najechaniu na dane klika w ikonę edycji
    \item Użytkownik wypełnia formularz
    \item Użytkownik klika w przycisk "Zmień hasło"
    \item Użytkownik otrzymuje informację zwrotną o statusie wykonanej operacji
\end{itemize}

\textbf{Przypadek użycia:} Dodanie profilu psa\\
\textbf{Aktor:} Niezalogowany użytkownik\\
\textbf{Opis:} Logowanie do serwisu\\
\textbf{Warunki wstępne:} Użytkownik niezalogowany, posiadający konto w serwisie\\
\textbf{Przebieg:}
\begin{itemize}
    \item Użytkownik wypełnia formularz logowania
    \item Użytkownik klika w przycisk "Zaloguj się"
    \begin{itemize}
        \item W przypadku sukcesu użytkownik zostaje przekierowany do strony aplikacji
        \item W przypadku błędu użytkownik jest informowany o błędzie
    \end{itemize}
\end{itemize}

\textbf{Przypadek użycia:} Logowanie\\
\textbf{Aktor:} Niezalogowany użytkownik\\
\textbf{Opis:} Logowanie do serwisu\\
\textbf{Warunki wstępne:} Użytkownik niezalogowany, posiadający konto w serwisie\\
\textbf{Przebieg:}
\begin{itemize}
    \item Użytkownik wypełnia formularz logowania
    \item Użytkownik klika w przycisk "Zaloguj się"
    \begin{itemize}
        \item W przypadku sukcesu użytkownik zostaje przekierowany do strony aplikacji
        \item W przypadku błędu użytkownik jest informowany o błędzie
    \end{itemize}
\end{itemize}

\textbf{Przypadek użycia:} Logowanie\\
\textbf{Aktor:} Niezalogowany użytkownik\\
\textbf{Opis:} Logowanie do serwisu\\
\textbf{Warunki wstępne:} Użytkownik niezalogowany, posiadający konto w serwisie\\
\textbf{Przebieg:}
\begin{itemize}
    \item Użytkownik wypełnia formularz logowania
    \item Użytkownik klika w przycisk "Zaloguj się"
    \begin{itemize}
        \item W przypadku sukcesu użytkownik zostaje przekierowany do strony aplikacji
        \item W przypadku błędu użytkownik jest informowany o błędzie
    \end{itemize}
\end{itemize}

\textbf{Przypadek użycia:} Logowanie\\
\textbf{Aktor:} Niezalogowany użytkownik\\
\textbf{Opis:} Logowanie do serwisu\\
\textbf{Warunki wstępne:} Użytkownik niezalogowany, posiadający konto w serwisie\\
\textbf{Przebieg:}
\begin{itemize}
    \item Użytkownik wypełnia formularz logowania
    \item Użytkownik klika w przycisk "Zaloguj się"
    \begin{itemize}
        \item W przypadku sukcesu użytkownik zostaje przekierowany do strony aplikacji
        \item W przypadku błędu użytkownik jest informowany o błędzie
    \end{itemize}
\end{itemize}

\textbf{Przypadek użycia:} Logowanie\\
\textbf{Aktor:} Niezalogowany użytkownik\\
\textbf{Opis:} Logowanie do serwisu\\
\textbf{Warunki wstępne:} Użytkownik niezalogowany, posiadający konto w serwisie\\
\textbf{Przebieg:}
\begin{itemize}
    \item Użytkownik wypełnia formularz logowania
    \item Użytkownik klika w przycisk "Zaloguj się"
    \begin{itemize}
        \item W przypadku sukcesu użytkownik zostaje przekierowany do strony aplikacji
        \item W przypadku błędu użytkownik jest informowany o błędzie
    \end{itemize}
\end{itemize}

\textbf{Przypadek użycia:} Logowanie\\
\textbf{Aktor:} Niezalogowany użytkownik\\
\textbf{Opis:} Logowanie do serwisu\\
\textbf{Warunki wstępne:} Użytkownik niezalogowany, posiadający konto w serwisie\\
\textbf{Przebieg:}
\begin{itemize}
    \item Użytkownik wypełnia formularz logowania
    \item Użytkownik klika w przycisk "Zaloguj się"
    \begin{itemize}
        \item W przypadku sukcesu użytkownik zostaje przekierowany do strony aplikacji
        \item W przypadku błędu użytkownik jest informowany o błędzie
    \end{itemize}
\end{itemize}

\textbf{Przypadek użycia:} Logowanie\\
\textbf{Aktor:} Niezalogowany użytkownik\\
\textbf{Opis:} Logowanie do serwisu\\
\textbf{Warunki wstępne:} Użytkownik niezalogowany, posiadający konto w serwisie\\
\textbf{Przebieg:}
\begin{itemize}
    \item Użytkownik wypełnia formularz logowania
    \item Użytkownik klika w przycisk "Zaloguj się"
    \begin{itemize}
        \item W przypadku sukcesu użytkownik zostaje przekierowany do strony aplikacji
        \item W przypadku błędu użytkownik jest informowany o błędzie
    \end{itemize}
\end{itemize}

\textbf{Przypadek użycia:} Logowanie\\
\textbf{Aktor:} Niezalogowany użytkownik\\
\textbf{Opis:} Logowanie do serwisu\\
\textbf{Warunki wstępne:} Użytkownik niezalogowany, posiadający konto w serwisie\\
\textbf{Przebieg:}
\begin{itemize}
    \item Użytkownik wypełnia formularz logowania
    \item Użytkownik klika w przycisk "Zaloguj się"
    \begin{itemize}
        \item W przypadku sukcesu użytkownik zostaje przekierowany do strony aplikacji
        \item W przypadku błędu użytkownik jest informowany o błędzie
    \end{itemize}
\end{itemize}

\textbf{Przypadek użycia:} Logowanie\\
\textbf{Aktor:} Niezalogowany użytkownik\\
\textbf{Opis:} Logowanie do serwisu\\
\textbf{Warunki wstępne:} Użytkownik niezalogowany, posiadający konto w serwisie\\
\textbf{Przebieg:}
\begin{itemize}
    \item Użytkownik wypełnia formularz logowania
    \item Użytkownik klika w przycisk "Zaloguj się"
    \begin{itemize}
        \item W przypadku sukcesu użytkownik zostaje przekierowany do strony aplikacji
        \item W przypadku błędu użytkownik jest informowany o błędzie
    \end{itemize}
\end{itemize}

\textbf{Przypadek użycia:} Logowanie\\
\textbf{Aktor:} Niezalogowany użytkownik\\
\textbf{Opis:} Logowanie do serwisu\\
\textbf{Warunki wstępne:} Użytkownik niezalogowany, posiadający konto w serwisie\\
\textbf{Przebieg:}
\begin{itemize}
    \item Użytkownik wypełnia formularz logowania
    \item Użytkownik klika w przycisk "Zaloguj się"
    \begin{itemize}
        \item W przypadku sukcesu użytkownik zostaje przekierowany do strony aplikacji
        \item W przypadku błędu użytkownik jest informowany o błędzie
    \end{itemize}
\end{itemize}

\textbf{Przypadek użycia:} Logowanie\\
\textbf{Aktor:} Niezalogowany użytkownik\\
\textbf{Opis:} Logowanie do serwisu\\
\textbf{Warunki wstępne:} Użytkownik niezalogowany, posiadający konto w serwisie\\
\textbf{Przebieg:}
\begin{itemize}
    \item Użytkownik wypełnia formularz logowania
    \item Użytkownik klika w przycisk "Zaloguj się"
    \begin{itemize}
        \item W przypadku sukcesu użytkownik zostaje przekierowany do strony aplikacji
        \item W przypadku błędu użytkownik jest informowany o błędzie
    \end{itemize}
\end{itemize}

\textbf{Przypadek użycia:} Logowanie\\
\textbf{Aktor:} Niezalogowany użytkownik\\
\textbf{Opis:} Logowanie do serwisu\\
\textbf{Warunki wstępne:} Użytkownik niezalogowany, posiadający konto w serwisie\\
\textbf{Przebieg:}
\begin{itemize}
    \item Użytkownik wypełnia formularz logowania
    \item Użytkownik klika w przycisk "Zaloguj się"
    \begin{itemize}
        \item W przypadku sukcesu użytkownik zostaje przekierowany do strony aplikacji
        \item W przypadku błędu użytkownik jest informowany o błędzie
    \end{itemize}
\end{itemize}

\textbf{Przypadek użycia:} Logowanie\\
\textbf{Aktor:} Niezalogowany użytkownik\\
\textbf{Opis:} Logowanie do serwisu\\
\textbf{Warunki wstępne:} Użytkownik niezalogowany, posiadający konto w serwisie\\
\textbf{Przebieg:}
\begin{itemize}
    \item Użytkownik wypełnia formularz logowania
    \item Użytkownik klika w przycisk "Zaloguj się"
    \begin{itemize}
        \item W przypadku sukcesu użytkownik zostaje przekierowany do strony aplikacji
        \item W przypadku błędu użytkownik jest informowany o błędzie
    \end{itemize}
\end{itemize}

\textbf{Przypadek użycia:} Logowanie\\
\textbf{Aktor:} Niezalogowany użytkownik\\
\textbf{Opis:} Logowanie do serwisu\\
\textbf{Warunki wstępne:} Użytkownik niezalogowany, posiadający konto w serwisie\\
\textbf{Przebieg:}
\begin{itemize}
    \item Użytkownik wypełnia formularz logowania
    \item Użytkownik klika w przycisk "Zaloguj się"
    \begin{itemize}
        \item W przypadku sukcesu użytkownik zostaje przekierowany do strony aplikacji
        \item W przypadku błędu użytkownik jest informowany o błędzie
    \end{itemize}
\end{itemize}

\textbf{Przypadek użycia:} Logowanie\\
\textbf{Aktor:} Niezalogowany użytkownik\\
\textbf{Opis:} Logowanie do serwisu\\
\textbf{Warunki wstępne:} Użytkownik niezalogowany, posiadający konto w serwisie\\
\textbf{Przebieg:}
\begin{itemize}
    \item Użytkownik wypełnia formularz logowania
    \item Użytkownik klika w przycisk "Zaloguj się"
    \begin{itemize}
        \item W przypadku sukcesu użytkownik zostaje przekierowany do strony aplikacji
        \item W przypadku błędu użytkownik jest informowany o błędzie
    \end{itemize}
\end{itemize}

\textbf{Przypadek użycia:} Logowanie\\
\textbf{Aktor:} Niezalogowany użytkownik\\
\textbf{Opis:} Logowanie do serwisu\\
\textbf{Warunki wstępne:} Użytkownik niezalogowany, posiadający konto w serwisie\\
\textbf{Przebieg:}
\begin{itemize}
    \item Użytkownik wypełnia formularz logowania
    \item Użytkownik klika w przycisk "Zaloguj się"
    \begin{itemize}
        \item W przypadku sukcesu użytkownik zostaje przekierowany do strony aplikacji
        \item W przypadku błędu użytkownik jest informowany o błędzie
    \end{itemize}
\end{itemize}

\textbf{Przypadek użycia:} Logowanie\\
\textbf{Aktor:} Niezalogowany użytkownik\\
\textbf{Opis:} Logowanie do serwisu\\
\textbf{Warunki wstępne:} Użytkownik niezalogowany, posiadający konto w serwisie\\
\textbf{Przebieg:}
\begin{itemize}
    \item Użytkownik wypełnia formularz logowania
    \item Użytkownik klika w przycisk "Zaloguj się"
    \begin{itemize}
        \item W przypadku sukcesu użytkownik zostaje przekierowany do strony aplikacji
        \item W przypadku błędu użytkownik jest informowany o błędzie
    \end{itemize}
\end{itemize}

\textbf{Przypadek użycia:} Logowanie\\
\textbf{Aktor:} Niezalogowany użytkownik\\
\textbf{Opis:} Logowanie do serwisu\\
\textbf{Warunki wstępne:} Użytkownik niezalogowany, posiadający konto w serwisie\\
\textbf{Przebieg:}
\begin{itemize}
    \item Użytkownik wypełnia formularz logowania
    \item Użytkownik klika w przycisk "Zaloguj się"
    \begin{itemize}
        \item W przypadku sukcesu użytkownik zostaje przekierowany do strony aplikacji
        \item W przypadku błędu użytkownik jest informowany o błędzie
    \end{itemize}
\end{itemize}

\textbf{Przypadek użycia:} Logowanie\\
\textbf{Aktor:} Niezalogowany użytkownik\\
\textbf{Opis:} Logowanie do serwisu\\
\textbf{Warunki wstępne:} Użytkownik niezalogowany, posiadający konto w serwisie\\
\textbf{Przebieg:}
\begin{itemize}
    \item Użytkownik wypełnia formularz logowania
    \item Użytkownik klika w przycisk "Zaloguj się"
    \begin{itemize}
        \item W przypadku sukcesu użytkownik zostaje przekierowany do strony aplikacji
        \item W przypadku błędu użytkownik jest informowany o błędzie
    \end{itemize}
\end{itemize}

\textbf{Przypadek użycia:} Logowanie\\
\textbf{Aktor:} Niezalogowany użytkownik\\
\textbf{Opis:} Logowanie do serwisu\\
\textbf{Warunki wstępne:} Użytkownik niezalogowany, posiadający konto w serwisie\\
\textbf{Przebieg:}
\begin{itemize}
    \item Użytkownik wypełnia formularz logowania
    \item Użytkownik klika w przycisk "Zaloguj się"
    \begin{itemize}
        \item W przypadku sukcesu użytkownik zostaje przekierowany do strony aplikacji
        \item W przypadku błędu użytkownik jest informowany o błędzie
    \end{itemize}
\end{itemize}

\textbf{Przypadek użycia:} Logowanie\\
\textbf{Aktor:} Niezalogowany użytkownik\\
\textbf{Opis:} Logowanie do serwisu\\
\textbf{Warunki wstępne:} Użytkownik niezalogowany, posiadający konto w serwisie\\
\textbf{Przebieg:}
\begin{itemize}
    \item Użytkownik wypełnia formularz logowania
    \item Użytkownik klika w przycisk "Zaloguj się"
    \begin{itemize}
        \item W przypadku sukcesu użytkownik zostaje przekierowany do strony aplikacji
        \item W przypadku błędu użytkownik jest informowany o błędzie
    \end{itemize}
\end{itemize}

\textbf{Przypadek użycia:} Logowanie\\
\textbf{Aktor:} Niezalogowany użytkownik\\
\textbf{Opis:} Logowanie do serwisu\\
\textbf{Warunki wstępne:} Użytkownik niezalogowany, posiadający konto w serwisie\\
\textbf{Przebieg:}
\begin{itemize}
    \item Użytkownik wypełnia formularz logowania
    \item Użytkownik klika w przycisk "Zaloguj się"
    \begin{itemize}
        \item W przypadku sukcesu użytkownik zostaje przekierowany do strony aplikacji
        \item W przypadku błędu użytkownik jest informowany o błędzie
    \end{itemize}
\end{itemize}

\textbf{Przypadek użycia:} Logowanie\\
\textbf{Aktor:} Niezalogowany użytkownik\\
\textbf{Opis:} Logowanie do serwisu\\
\textbf{Warunki wstępne:} Użytkownik niezalogowany, posiadający konto w serwisie\\
\textbf{Przebieg:}
\begin{itemize}
    \item Użytkownik wypełnia formularz logowania
    \item Użytkownik klika w przycisk "Zaloguj się"
    \begin{itemize}
        \item W przypadku sukcesu użytkownik zostaje przekierowany do strony aplikacji
        \item W przypadku błędu użytkownik jest informowany o błędzie
    \end{itemize}
\end{itemize}

\textbf{Przypadek użycia:} Logowanie\\
\textbf{Aktor:} Niezalogowany użytkownik\\
\textbf{Opis:} Logowanie do serwisu\\
\textbf{Warunki wstępne:} Użytkownik niezalogowany, posiadający konto w serwisie\\
\textbf{Przebieg:}
\begin{itemize}
    \item Użytkownik wypełnia formularz logowania
    \item Użytkownik klika w przycisk "Zaloguj się"
    \begin{itemize}
        \item W przypadku sukcesu użytkownik zostaje przekierowany do strony aplikacji
        \item W przypadku błędu użytkownik jest informowany o błędzie
    \end{itemize}
\end{itemize}

\textbf{Przypadek użycia:} Logowanie\\
\textbf{Aktor:} Niezalogowany użytkownik\\
\textbf{Opis:} Logowanie do serwisu\\
\textbf{Warunki wstępne:} Użytkownik niezalogowany, posiadający konto w serwisie\\
\textbf{Przebieg:}
\begin{itemize}
    \item Użytkownik wypełnia formularz logowania
    \item Użytkownik klika w przycisk "Zaloguj się"
    \begin{itemize}
        \item W przypadku sukcesu użytkownik zostaje przekierowany do strony aplikacji
        \item W przypadku błędu użytkownik jest informowany o błędzie
    \end{itemize}
\end{itemize}

\textbf{Przypadek użycia:} Logowanie\\
\textbf{Aktor:} Niezalogowany użytkownik\\
\textbf{Opis:} Logowanie do serwisu\\
\textbf{Warunki wstępne:} Użytkownik niezalogowany, posiadający konto w serwisie\\
\textbf{Przebieg:}
\begin{itemize}
    \item Użytkownik wypełnia formularz logowania
    \item Użytkownik klika w przycisk "Zaloguj się"
    \begin{itemize}
        \item W przypadku sukcesu użytkownik zostaje przekierowany do strony aplikacji
        \item W przypadku błędu użytkownik jest informowany o błędzie
    \end{itemize}
\end{itemize}

\textbf{Przypadek użycia:} Logowanie\\
\textbf{Aktor:} Niezalogowany użytkownik\\
\textbf{Opis:} Logowanie do serwisu\\
\textbf{Warunki wstępne:} Użytkownik niezalogowany, posiadający konto w serwisie\\
\textbf{Przebieg:}
\begin{itemize}
    \item Użytkownik wypełnia formularz logowania
    \item Użytkownik klika w przycisk "Zaloguj się"
    \begin{itemize}
        \item W przypadku sukcesu użytkownik zostaje przekierowany do strony aplikacji
        \item W przypadku błędu użytkownik jest informowany o błędzie
    \end{itemize}
\end{itemize}

\textbf{Przypadek użycia:} Logowanie\\
\textbf{Aktor:} Niezalogowany użytkownik\\
\textbf{Opis:} Logowanie do serwisu\\
\textbf{Warunki wstępne:} Użytkownik niezalogowany, posiadający konto w serwisie\\
\textbf{Przebieg:}
\begin{itemize}
    \item Użytkownik wypełnia formularz logowania
    \item Użytkownik klika w przycisk "Zaloguj się"
    \begin{itemize}
        \item W przypadku sukcesu użytkownik zostaje przekierowany do strony aplikacji
        \item W przypadku błędu użytkownik jest informowany o błędzie
    \end{itemize}
\end{itemize}

\textbf{Przypadek użycia:} Logowanie\\
\textbf{Aktor:} Niezalogowany użytkownik\\
\textbf{Opis:} Logowanie do serwisu\\
\textbf{Warunki wstępne:} Użytkownik niezalogowany, posiadający konto w serwisie\\
\textbf{Przebieg:}
\begin{itemize}
    \item Użytkownik wypełnia formularz logowania
    \item Użytkownik klika w przycisk "Zaloguj się"
    \begin{itemize}
        \item W przypadku sukcesu użytkownik zostaje przekierowany do strony aplikacji
        \item W przypadku błędu użytkownik jest informowany o błędzie
    \end{itemize}
\end{itemize}

% \newline
% \textbf{Przypadek użycia:} \\
% \textbf{Aktor:} \\
% \textbf{Opis:} \\
% \textbf{Warunki wstępne:} \\
% \textbf{Przebieg:} \\
% \begin{itemize}
%     \item 
% \end{itemize}

\section{Widoki, mockupy}

\section{Backend}

\section{Frontend}